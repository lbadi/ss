\documentclass[hyperref={pdfpagelayout=SinglePage}]{beamer}

%Template
\usetheme{Warsaw}
\usecolortheme{default}
\usefonttheme[onlymath]{serif}

%Packages
\usepackage[utf8]{inputenc}
\usepackage[spanish,activeacute]{babel}
\usepackage{lipsum}
\usepackage{graphicx}
\usepackage{fancyhdr}
\usepackage{float}
\usepackage{adjustbox}
\usepackage{subfigure}
\usepackage{amsmath}
\usepackage{ragged2e}
\usepackage{color}
\usepackage{listings}
\usepackage{animate}
\usepackage{hyperref}

%Code
\lstset{ %
basicstyle=\small,       % the size of the fonts that are used for the code
numbers=none,                   % where to put the line-numbers
numberstyle=\footnotesize,      % the size of the fonts that are used for the line-numbers
stepnumber=1,                   % the step between two line-numbers. If it is 1 each line will be numbered
numbersep=5pt,                  % how far the line-numbers are from the code
backgroundcolor=\color{white},  % choose the background color. You must add \usepackage{color}
showspaces=false,               % show spaces adding particular underscores
showstringspaces=false,         % underline spaces within strings
showtabs=false,                 % show tabs within strings adding particular underscores
frame=single,           % adds a frame around the code
tabsize=4,          % sets default tabsize to N spaces
captionpos=b,           % sets the caption-position to bottom
breaklines=true,        % sets automatic line breaking
breakatwhitespace=false,    % sets if automatic breaks should only happen at whitespace
escapeinside={\%*}{*)}          % if you want to add a comment within your code
}

%Captions
\renewcommand{\lstlistingname}{Código}
\renewcommand\spanishtablename{Tabla}
\renewcommand{\figurename}{Gráfico}

%More
\makeatletter
\@addtoreset{subfigure}{framenumber}
\makeatother

\expandafter\def\expandafter\insertshorttitle\expandafter{%
  \insertshorttitle\hfill%
  \insertframenumber\,/\,\inserttotalframenumber}

\newcommand\Wider[2][5em]{%
\makebox[\linewidth][c]{%
  \begin{minipage}{\dimexpr\textwidth+#1\relax}
  \raggedright#2
  \end{minipage}%
  }%
}

%List of parts
\makeatletter
\AtBeginPart{%
    \addtocontents{parttoc}{\protect\beamer@partintoc{\the\c@part}{\beamer@partnameshort}{\the\c@page}}%
    \frame{\partpage}%
}
\newcommand{\parttableofcontents}{\@starttoc{parttoc}}
\newcommand{\beamer@partintoc}[3]{#2\par}
\makeatother

%Document Data
\title{Dinámica Peatonal}
\subtitle{Trabajo Práctico Nro. 6}
\author{Badi Leonel, Buchhalter Nicolás Demián y Meola Franco Román}
\subject{Simulación de Sistemas}
\date{\today}

\begin{document}

\begin{frame}[plain]
    \frametitle{} 
    \titlepage
    \centering
	Grupo 3
\end{frame}

%Part List
%\begin{frame}
%\frametitle{Que vamos a ver}
%\parttableofcontents
%\end{frame}

%First Part
%\part{Egreso de una habitación}

\section{Fundamentos}

\subsection{Introducción}

\begin{frame}
\frametitle{Fundamentos}
\framesubtitle{Introducción}
\begin{itemize}
	\item Vamos a simular el egreso de personas de una habitación
	\item Para ello utilizaremos el \textit{Contractile Particle Model} 
	\item No se utilizaron plataformas de simulación peatonal 
\end{itemize}
\end{frame}

\subsection{Variables relevantes}

\section{Implementación}

\subsection{Generación de las partículas}

\begin{frame}
\frametitle{Implementación}
\framesubtitle{Generación de las partículas}
\begin{itemize}
	\item Posiciones $(x,y)$ aleatorias
	\begin{itemize}
		\item Verificando que no se superpongan
		\item Intentando hasta 10000 veces por partícula para obtener una ubicación válida 
	\end{itemize}
	\item $N = 300$
	\item $r_{min} = 0.15m$
	\item $r_{max} = 0.32m$
	\item $v_{d}^{max} = 1.55 \frac{m}{s}$
\end{itemize}
\end{frame}

\subsection{Simulación}

\begin{frame}
\frametitle{Simulación}
\framesubtitle{Variables relevantes del sistema}
\begin{itemize}
	\item \textbf{Parámetros de la habitación:}
	\begin{itemize}
		\item Habitación cuadrada de $20m$ x $20m$
		\item Puerta central de $1.2m$ de apertura
	\end{itemize}
	\item \textbf{Parámetros del \textit{Contractile Particle Model}:}
	\begin{itemize}
		\item $\tau = 0.5s$
		\item $\beta = 0.9$
	\end{itemize}
\end{itemize}
\end{frame}

\begin{frame}
\frametitle{Simulación}
\framesubtitle{Variables relevantes de la simulación}
\begin{itemize}
	\item $t$ $[s]$: tiempo en segundos a visualizar
	\item $dt$ $[s]$: tiempo en segundos del paso de la simulación
	\item $k$: relación entre cantidad de pasos simulados y escritos
\end{itemize}
\end{frame}

\begin{frame}[fragile]
\frametitle{Simulación}
\framesubtitle{Algoritmo de simulación}
\begin{lstlisting}[language=Java, caption = Algoritmo de simulación]
public void simulate(double t, double dt, int k){
	writeFrame(0);
	int framesWrited = 1;
	double totalTimeSimulated = 0;
	moveSystem(dt);
	totalTimeSimulated += dt;
	while(totalTimeSimulated < t){
		for(int i = 0; i < k; i++){
			moveSystem(dt);
			totalTimeSimulated += dt;
		}
		writeFrame(framesWrited++);
	}
}
\end{lstlisting}
\end{frame}

\begin{frame}
\frametitle{Simulación}
\framesubtitle{Detalles de implementación}
\begin{itemize}
	\item Para los valores de $N$ que vamos a analizar se obtuvieron tiempos de procesamiento muy bajos
	\item Por lo tanto no se necesitó contar con el \textit{Cell Index Method}
	\item Se implementó un modelo que contempla una lista de \textit{targets} para cada una de las partículas
\end{itemize}
\end{frame}

\subsection{Visualización}

\begin{frame}
\frametitle{Implementación}
\framesubtitle{Visualización}
\begin{itemize}
	\item La simulación y la visualización son independientes
	\item El algoritmo de simulación escribe un archivo \texttt{.tsv} con los siguientes datos:
	\begin{itemize}
		\item $(x,y)$
		\item $r$
		\item Color RGB para indicar las velocidades, donde R es la componente en el eje Y y G es la componente en eje X
	\end{itemize}
	\item Por último, se carga en \texttt{Ovito} el archivo de salida\texttt{.tsv} para realizar la visualización
\end{itemize}
\end{frame}

\section{Resultados}

\subsection{Gráficos y Tablas}

\begin{frame}
\frametitle{Resultados}
\framesubtitle{Variables relevantes}
\begin{itemize}
	\item \textbf{Parámetros de la simulación:}
	\begin{itemize}
		\item $k = 5$
		\item $dt = 0.05s$
	\end{itemize}
\end{itemize}
\end{frame}

%\begin{frame}
%\frametitle{Resultados}
%\framesubtitle{? para distintos valores de ?}
%\begin{center}
%\begin{table}[h]
%\centering
%\adjustbox{max height=\dimexpr\textheight-3.0cm\relax,
%           max width=\textwidth}{
%\begin{tabular}{cc}
%\hline
%\textbf{?} & \textbf{?}\\ \hline
%?&?\\ 
%\end{tabular}
%}
%\caption{? para distintos valores de ?}
%\end{table}
%\end{center}
%\end{frame}

\begin{frame}
\Wider{
\frametitle{Gráfico de número de evacuados en función del tiempo}
\framesubtitle{Para $N = 300$, $r_{min} = 0.15m$, $r_{max} = 0.32m$ y $v_{d}^{max} = 1.55 \frac{m}{s}$}
\begin{figure}[H]
	\centering
    \includegraphics[width=\textwidth]{charts/ejA.png}
\end{figure}
}
\end{frame}

\begin{frame}
\Wider{
\frametitle{Gráfico del comportamiento promedio del sistema}
\framesubtitle{Para $N = 300$, $r_{min} = 0.15m$, $r_{max} = 0.32m$ y $v_{d}^{max} = 1.55 \frac{m}{s}$}
\begin{figure}[H]
	\centering
    \includegraphics[width=\textwidth]{charts/ejB.png}
\end{figure}
}
\end{frame}

\begin{frame}
\Wider{
\frametitle{Gráfico de la evolución del caudal en función del tiempo}
\framesubtitle{Para $N = 300$, $r_{min} = 0.15m$, $r_{max} = 0.32m$, $v_{d}^{max} = 1.55 \frac{m}{s}$ y Ventana = $10s$}
\begin{figure}[H]
	\centering
    \includegraphics[width=\textwidth]{charts/ejC-10.png}
\end{figure}
}
\end{frame}

\begin{frame}
\Wider{
\frametitle{Gráfico de $v_{d}^{max}$ en función del tiempo de evacuación}
\framesubtitle{Para $N = 300$, $r_{min} = 0.15m$, $r_{max} = 0.32m$ y 3 simulaciones en cada caso}
\begin{figure}[H]
	\centering
    \includegraphics[width=\textwidth]{charts/ejD.png}
\end{figure}
}
\end{frame}

\subsection{Animaciones}

\begin{frame}[plain]
\frametitle{Animación de la simulación}
\framesubtitle{Para $N = 300$, $r_{min} = 0.15m$, $r_{max} = 0.32m$ y $v_{d}^{max} = 1.55 \frac{m}{s}$}
\begin{figure}[H]
	\centering
	\animategraphics[loop,controls,width=0.85\textwidth]{10}{animation/300n}{00001}{00242}
\end{figure}
\end{frame}

\subsection{Conclusiones}

\begin{frame}
\frametitle{Conclusiones}
\begin{itemize}
	\item Se logró representar el movimiento de personas en estado no competitivo usando el \textit{Contractile Particle Model}
	\item A medida que se aumenta la velocidad, se puede ver como disminuye el tiempo de salida. En este caso no se obtuvo \textit{faster is slower}. Una de las posibles causas de esto es que el movimiento de los agentes es no competitivo.
	\item Implementación simple donde cada partícula o agente es independiente. 
\end{itemize}	
\end{frame} 

\begin{frame}[plain]
    \centering
	\Huge Gracias
\end{frame}

\end{document}
