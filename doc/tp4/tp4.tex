\documentclass[hyperref={pdfpagelayout=SinglePage}]{beamer}

%Template
\usetheme{Warsaw}
\usecolortheme{default}
\usefonttheme[onlymath]{serif}

%Packages
\usepackage[utf8]{inputenc}
\usepackage[spanish,activeacute]{babel}
\usepackage{lipsum}
\usepackage{graphicx}
\usepackage{fancyhdr}
\usepackage{float}
\usepackage{adjustbox}
\usepackage{subfigure}
\usepackage{amsmath}
\usepackage{ragged2e}
\usepackage{color}
\usepackage{listings}
\usepackage{animate}

%Code
\lstset{ %
basicstyle=\small,       % the size of the fonts that are used for the code
numbers=none,                   % where to put the line-numbers
numberstyle=\footnotesize,      % the size of the fonts that are used for the line-numbers
stepnumber=1,                   % the step between two line-numbers. If it is 1 each line will be numbered
numbersep=5pt,                  % how far the line-numbers are from the code
backgroundcolor=\color{white},  % choose the background color. You must add \usepackage{color}
showspaces=false,               % show spaces adding particular underscores
showstringspaces=false,         % underline spaces within strings
showtabs=false,                 % show tabs within strings adding particular underscores
frame=single,           % adds a frame around the code
tabsize=4,          % sets default tabsize to N spaces
captionpos=b,           % sets the caption-position to bottom
breaklines=true,        % sets automatic line breaking
breakatwhitespace=false,    % sets if automatic breaks should only happen at whitespace
escapeinside={\%*}{*)}          % if you want to add a comment within your code
}

%Captions
\renewcommand{\lstlistingname}{Código}
\renewcommand\spanishtablename{Tabla}
\renewcommand{\figurename}{Gráfico}

%More
\makeatletter
\@addtoreset{subfigure}{framenumber}
\makeatother

\expandafter\def\expandafter\insertshorttitle\expandafter{%
  \insertshorttitle\hfill%
  \insertframenumber\,/\,\inserttotalframenumber}

\newcommand\Wider[2][5em]{%
\makebox[\linewidth][c]{%
  \begin{minipage}{\dimexpr\textwidth+#1\relax}
  \raggedright#2
  \end{minipage}%
  }%
}

%List of parts
\makeatletter
\AtBeginPart{%
    \addtocontents{parttoc}{\protect\beamer@partintoc{\the\c@part}{\beamer@partnameshort}{\the\c@page}}%
    \frame{\partpage}%
}
\newcommand{\parttableofcontents}{\@starttoc{parttoc}}
\newcommand{\beamer@partintoc}[3]{#2\par}
\makeatother

%Document Data
\title{Dinámica Molecular regida por el paso temporal}
\subtitle{Trabajo Práctico Nro. 4}
\author{Badi Leonel, Buchhalter Nicolás Demián y Meola Franco Román}
\subject{Simulación de Sistemas}
\date{\today}

\begin{document}

\begin{frame}
    \frametitle{} 
    \titlepage
    \centering
	Grupo 3
\end{frame}

%Part List
\frame{\frametitle{Que vamos a ver}\parttableofcontents}
    
\part{Oscilador Puntual Amortiguado}
    
\section{Fundamentos}

\subsection{Introducción}

\begin{frame}
\frametitle{Fundamentos}
\framesubtitle{Introducción}
\begin{itemize}
	\item Vamos a comparar los errores cometidos por distintos sistemas de integración
	\item Oscilador amortiguado: Sistema con sólo una partícula puntual cuya solución analítica es conocida
	\item Se implementaron:
	\begin{itemize}
		\item \textit{Beeman}
		\item \textit{Velocity Verlet}
		\item \textit{Gear Predictor Corrector de orden 5}
	\end{itemize}
\end{itemize}
\end{frame}

\subsection{Variables relevantes}

\begin{frame}
\frametitle{Fundamentos}
\framesubtitle{Variables relevantes}
\begin{itemize}
\item Parámetros
	\begin{itemize}
		\item $m = 70$
		\item $k = 10000$
		\item $\gamma = 100$
		\item $t_{f} = 5$ 
	\end{itemize}
\item Condiciones iniciales
	\begin{itemize}
		\item $r(t=0) = 1$
		\item $v(t=0) = -\frac{2\gamma}{m}$
	\end{itemize}
\end{itemize}
\end{frame}

\section{Implementación}

\subsection{Simulación}

\begin{frame}[fragile]
\frametitle{Implementación}
\framesubtitle{Cálculo Numérico}
\begin{lstlisting}[language=Java, caption = Método de Gear Predictor Corrector.]
void simulateGear(double time, double deltaT) {
	double simTime = 0;
    Oscilator oscilator = new Oscilator();
	oscilator.writePositionAndError();
    oscilator.makeEulerStep(deltaT);
    simTime += deltaT;
	oscilator.writePositionAndError();
    while (simTime < time) {
    	oscilator.makeGearStep(deltaT);
        simTime += deltaT;
		oscilator.writePositionAndError();
    }
}
\end{lstlisting}
\end{frame}

\begin{frame}
\frametitle{Implementación}
\framesubtitle{Detalles de precisión}
\begin{itemize}
	\item Todas las operaciones se realizan en \texttt{double}
	\item Se utilizan cinco cifras decimales como \texttt{output} en los archivos de salida de resultados y errores.
\end{itemize}
\end{frame}

\section{Resultados}

\subsection{Tablas}

\begin{frame}
\frametitle{Resultados}
\framesubtitle{Error total normalizado por el número total de pasos para distintos valores de $\Delta t$}
\begin{center}
\begin{table}[h]
\centering
\adjustbox{max height=\dimexpr\textheight-3.0cm\relax,
           max width=\textwidth}{
\begin{tabular}{ccc}
\hline
\textbf{$\Delta t$} & \textbf{Método} & \textbf{$E$}\\ \hline
0.01&\textit{Beeman}&0,00471\\
0.01&\textit{Verlet}&0,00663\\
0.01&\textit{Gear}&0,33624\\
0.001&\textit{Beeman}&0,00235\\
0.001&\textit{Verlet}&0,00225\\
\textbf{0.001}&\textbf{\textit{Gear}}&\textbf{-0,00199}\\
0.0001&\textit{Beeman}&0,00225\\
0.0001&\textit{Verlet}&0,00224\\
0.0001&\textit{Gear}&0,00228\\
\end{tabular}
}
\caption{Suma de las diferencias al cuadrado para todos los pasos temporales normalizado por el número total de pasos}
\end{table}
\end{center}
\end{frame}

\subsection{Gráficos}

\begin{frame}
\Wider{
\frametitle{Resultados}
\framesubtitle{Gráfico de $x(t)$ para el oscilador puntual amortiguado con $\Delta t = 0.01$}
\begin{figure}[H]
        \centering
        \includegraphics[width=\textwidth]{{images/0.01}.png}
\end{figure}
}
\end{frame}

\begin{frame}
\Wider{
\frametitle{Resultados}
\framesubtitle{Gráfico de $x(t)$ para el oscilador puntual amortiguado con $\Delta t = 0.001$}
\begin{figure}[H]
        \centering
        \includegraphics[width=\textwidth]{{images/0.001}.png}
\end{figure}
}
\end{frame}

\begin{frame}
\Wider{
\frametitle{Resultados}
\framesubtitle{Gráfico de $E$ para el oscilador puntual amortiguado con $\Delta t = 0.01$}
\begin{figure}[H]
        \centering
        \includegraphics[width=\textwidth]{{images/Error-0.01}.png}
\end{figure}
}
\end{frame}

\begin{frame}
\Wider{
\frametitle{Resultados}
\framesubtitle{Gráfico de $E$ para el oscilador puntual amortiguado con $\Delta t = 0.001$}
\begin{figure}[H]
        \centering
        \includegraphics[width=\textwidth]{{images/Error-0.001}.png}
\end{figure}
}
\end{frame}

\subsection{Conclusiones}

\begin{frame}
\frametitle{Conclusiones}
\begin{itemize}
	\item Para una cantidad de pasos baja (500 pasos, $\Delta t = 0.01$), el error de \textit{Gear Predictor-Corrector} aumenta, simulando un oscilador no amortiguado.
	\item Con un $\Delta t = 0.001$ obtuvimos resultados con errores muy bajos para los tres métodos.	
	\item Con 50000 pasos ($\Delta t = 0.0001$), los tres métodos tienen un error que varía recíen en la quinta cifra decimal.
	\item El esquema de integración que mejor resulta para este sistema es \textit{Gear Predictor-Corrector} para $\Delta t = 0.001$, es decir, 5000 pasos.
	\end{itemize}	
\end{frame}
    
\part{Formación del Sistema Solar}
    
\section{Fundamentos}

\subsection{Introducción}

\begin{frame}
\frametitle{Fundamentos}
\framesubtitle{Introducción}
\begin{itemize}
	\item Usando el esquema de integración de \textit{Beeman} vamos a simular el nacimiento del sistema solar.
	\item Se simularán $N$ partículas que orbiten alrededor del Sol.
	\item Las partículas se irán agrupando en planetas a medida que el sistema evolucione.
\end{itemize}
\end{frame}

\subsection{Variables relevantes}

\section{Implementación}

\subsection{Generación de los agentes}

\begin{frame}
\frametitle{Implementación}
\framesubtitle{Generación de los agentes}
\begin{itemize}
	\item Posiciones $(x,y)$ aleatorias para todas las partículas
	\item $v_{0t}$ tal que todas las partículas tengan el mismo $L$
	\item $v_{0n} = 0$.
	\item Distancia al sol entre 10 a la 9 y 10 a la 10
	\item Angulo respecto al Sol entre 0 y 2 pi
\end{itemize}
\end{frame}

\subsection{Simulación}

\begin{frame}
\frametitle{Simulación}
\framesubtitle{Variables relevantes}
\begin{itemize}
	\item $\Delta t$: cantidad de pasos.
	\item $k$ relación entre cantidad de pasos simulados y visualizados.
	\item \texttt{time}: Tiempo en segundos a visualizar
\end{itemize}
\end{frame}

\begin{frame}
\frametitle{Simulación}
\framesubtitle{Detalles de implementación}
\begin{itemize}
	\item utilización del cell index method del tp anterior para calcular las colisiones de las partículas.
	%TODO 
	%\item Se modificó el cell index method para que soporte radios de interacción distintos para cada una de las particulas
	\item Para las partículas que se alejen más de 2 x 10 a la 4 no las consideramos dentro del sistema.
	\item Para simplificar, luego de la colisión de dos partículas, se obtiene una nueva con un radio correspondiente a la suma de los radios de las dos. 
\end{itemize}
\end{frame}

\begin{frame}
\frametitle{Simulación}
\framesubtitle{Problemas encontrados}
\begin{itemize}
	\item Manejo numérico de grandes dimensiones.
	\begin{itemize}
		\item Para poder mantener en memoria números tan grandes, utilizando la precisión double .
		\item Se normalizó la distancia por 10 a la 6
		\item Se normalizó la masa por 10 a la 25
	\end{itemize}
	\item El radio de las partículas en comparación con las dimensiones del sistema solar era muy chico
	\begin{itemize}
			\item Dificultaba la visualización, sobre todo para una gran cantidad de partículas.
			\item El radio de interacción es diferente al radio de visualización.
	\end{itemize}
\end{itemize}
\end{frame}

\subsection{Visualización}

\begin{frame}[fragile]
\frametitle{Implementación}
\framesubtitle{Cálculo Numérico}
\begin{lstlisting}[language=Java, caption = Método de Gear Predictor Corrector.]
void simulateGear(double time, double deltaT) {
	double simTime = 0;
    Oscilator oscilator = new Oscilator();
	oscilator.writePositionAndError();
    oscilator.makeEulerStep(deltaT);
    simTime += deltaT;
	oscilator.writePositionAndError();
    while (simTime < time) {
    	oscilator.makeGearStep(deltaT);
        simTime += deltaT;
		oscilator.writePositionAndError();
    }
}
\end{lstlisting}
\end{frame}

\begin{frame}
\frametitle{Implementación}
\framesubtitle{Visualización}
\begin{itemize}
	\item La simulación y la visualización son independientes
	\item El algoritmo de simulación escribe un archivo \texttt{.tsv} con los siguientes datos:
\begin{itemize}
\item $(x,y)$
\item $r$
\item Color RGB para indicar las velocidades, donde R es la componente en el eje Y y G es la componente en eje X
\end{itemize}
\item Por último, se carga en \texttt{Ovito} el archivo de salida\texttt{.tsv} para realizar la visualización
\end{itemize}
\end{frame}


\section{Resultados}

\subsection{Gráficos}

\begin{frame}
\Wider{
\frametitle{Resultados}
\framesubtitle{Gráfico de las energías $U, K y E_{T}$ para la simulación de $N = 100$}
\begin{figure}[H]
	\centering
    \includegraphics[width=\textwidth]{example-image}
\end{figure}
}
\end{frame}

\begin{frame}
\Wider{
\frametitle{Resultados}
\framesubtitle{Gráfico de las energías $U, K y E_{T}$ para la simulación de $N = 1000$}
\begin{figure}[H]
	\centering
    \includegraphics[width=\textwidth]{example-image}
\end{figure}
}
\end{frame}

\begin{frame}
\Wider{
\frametitle{Resultados}
\framesubtitle{Gráfico de las energías $U, K y E_{T}$ para la simulación de $N = 10000$}
\begin{figure}[H]
	\centering
    \includegraphics[width=\textwidth]{example-image}
\end{figure}
}
\end{frame}

\subsection{Animaciones}

\begin{frame}
\frametitle{Resultados}
\framesubtitle{Animación de la simulación para $N = 100$}
\begin{figure}[H]
	\centering
	\includegraphics[width=0.75\textheight]{example-image}
%	\animategraphics[loop,controls,width=0.75\textheight]{10}{animation/dmer-100-}{0000}{0200}
\end{figure}
\end{frame}

\begin{frame}
\frametitle{Resultados}
\framesubtitle{Animación de la simulación para $N = 1000$}
\begin{figure}[H]
	\centering
	\includegraphics[width=0.75\textheight]{example-image}
%	\animategraphics[loop,controls,width=0.75\textheight]{10}{animation/dmer-100-}{0000}{0200}
\end{figure}
\end{frame}

\begin{frame}
\frametitle{Resultados}
\framesubtitle{Animación de la simulación para $N = 10000$}
\begin{figure}[H]
	\centering
	\includegraphics[width=0.75\textheight]{example-image}
%	\animategraphics[loop,controls,width=0.75\textheight]{10}{animation/dmer-100-}{0000}{0200}
\end{figure}
\end{frame}

\begin{frame}
\frametitle{Resultados}
\framesubtitle{Animación de la simulación para $N = 50000$}
\begin{figure}[H]
	\centering
	\includegraphics[width=0.75\textheight]{example-image}
%	\animategraphics[loop,controls,width=0.75\textheight]{10}{animation/dmer-100-}{0000}{0200}
\end{figure}
\end{frame}

\subsection{Conclusiones}

\begin{frame}
\frametitle{Conclusiones}
\begin{itemize}
	\item El paso temporal ($\Delta t$) óptimo para simular el sistema es ?.
	\end{itemize}	
\end{frame}

\begin{frame}[plain,c]
\begin{center}
	\Huge Gracias
\end{center}
\end{frame}
    
\end{document}
