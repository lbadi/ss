\documentclass[hyperref={pdfpagelayout=SinglePage}]{beamer}
\usetheme{Warsaw}
\usecolortheme{default}
\usefonttheme[onlymath]{serif}
\usepackage[utf8]{inputenc}
\usepackage[spanish,activeacute]{babel}

\usepackage{graphicx}
\usepackage{fancyhdr}
\usepackage{float}
\usepackage{adjustbox}
\usepackage{subfigure}
\usepackage{amsmath}
\usepackage{ragged2e}

\usepackage{color}
\usepackage{listings}
\lstset{ %
basicstyle=\footnotesize,       % the size of the fonts that are used for the code
numbers=none,                   % where to put the line-numbers
numberstyle=\footnotesize,      % the size of the fonts that are used for the line-numbers
stepnumber=1,                   % the step between two line-numbers. If it is 1 each line will be numbered
numbersep=10pt,                  % how far the line-numbers are from the code
backgroundcolor=\color{white},  % choose the background color. You must add \usepackage{color}
showspaces=false,               % show spaces adding particular underscores
showstringspaces=false,         % underline spaces within strings
showtabs=false,                 % show tabs within strings adding particular underscores
frame=single,           % adds a frame around the code
tabsize=4,          % sets default tabsize to 2 spaces
captionpos=b,           % sets the caption-position to bottom
breaklines=true,        % sets automatic line breaking
breakatwhitespace=false,    % sets if automatic breaks should only happen at whitespace
escapeinside={\%*}{*)}          % if you want to add a comment within your code
}
\renewcommand{\lstlistingname}{Salida}
\renewcommand\spanishtablename{Tabla}

\expandafter\def\expandafter\insertshorttitle\expandafter{%
  \insertshorttitle\hfill%
  \insertframenumber\,/\,\inserttotalframenumber}

\title{Autómatas Celulares}
\subtitle{Trabajo Practico Nro. 2}
%\author{Grupo 3}
\author{Badi Leonel, Buchhalter Nicolás Demián y Meola Franco Román}
\subject{Simulación de Sistemas}
\date{\today}

\makeatletter
\@addtoreset{subfigure}{framenumber}
\makeatother

\begin{document}

\renewcommand{\figurename}{Grafico}

\begin{frame}[plain]
    \frametitle{} 
    \titlepage
\end{frame}

\section{Fundamentos}

\subsection{Introducción}

\begin{frame}
\frametitle{Fundamentos}
\framesubtitle{Introducción}
\begin{itemize}
	\item Autómata Off-Latice de Bandadas de Agentes Autopropulsados.
	\item Basado en el algoritmo del trabajo de \textit{Novel type of phase transition in a system of self-driven particles}.
	\item Cada agente será representado por un vector de velocidad cuyo origen estará ubicado en la posición de la partícula para cada tiempo de la simulación $t$.
\end{itemize}
\end{frame}

\subsection{Variables relevantes}

\begin{frame}
\frametitle{Fundamentos}
\framesubtitle{Variables relevantes}
\begin{itemize}
	\item $N$ : cantidad de agentes
	\item $L$ : longitud del lado del área de simulación
	\item $v$ : módulo de la velocidad
	\item $\rho = \frac{N}{L^2}$ : densidad
	\item $\eta$ : amplitud del ruido
	\item $v_{a}$ : parámetro de orden
	\item Nos interesa obtener las curvas de  $v_{a}$ en función de $\eta$ y $\rho$.
\end{itemize}
\end{frame}

\section{Implementación}

\subsection{Generación de los agentes}

\begin{frame}
\frametitle{Implementación}
\framesubtitle{Generación de los agentes}
\begin{itemize}
	\item Posiciones $(x,y)$ aleatorias para cada agente.
	\item $r_{c} = 0.5$ (radio de interacción de las partículas)
	\item $v_{cte} = 0.3$
	\item $\theta_{0} = rand() * 2\pi$ (ángulo inicial aleatorio)
	\item $\delta_{\theta} = {(rand() * \eta) - \frac{\eta}{2}} $
\end{itemize}
\end{frame}

\subsection{Algoritmo}

\begin{frame}[fragile]
\frametitle{Implementación}
\framesubtitle{Algoritmo}

\begin{lstlisting}[language=Java, caption = Código]
public static void main(String[] args) {
	log.info("Hola Mundo!");
	(...)
}
\end{lstlisting}


\end{frame}

\section{Resultados}

\subsection{Tablas}

\begin{frame}
\frametitle{Resultados}
\framesubtitle{Tabla de la prueba $N = 10$}
\begin{center}
\begin{table}[h]
\centering
\adjustbox{max height=\dimexpr\textheight-3.0cm\relax,
           max width=\textwidth}{
\begin{tabular}{ccc}
\hline
\textbf{$\eta$} & \textbf{$v_{a}$} & \textbf{$t(s)$} \\ \hline
0.3&0.90837&0\\
0.6&0.98470&0\\
0.9&0.70629&0\\
1.2&0.83263&0\\
1.5&0.20089&0\\
1.8&0.30686&0\\
2.1&0.36970&0\\
2.4&0.21550&0\\
\end{tabular}

}
\caption{Datos de la curva de $v_{a}$ para $N = 10$.}
\end{table}
\end{center}
\end{frame}

\begin{frame}
\frametitle{Resultados}
\framesubtitle{Tabla de la prueba $N = 50$}
\begin{center}
\begin{table}[h]
\centering
\adjustbox{max height=\dimexpr\textheight-3.0cm\relax,
           max width=\textwidth}{
\begin{tabular}{ccc}
\hline
\textbf{$\eta$} & \textbf{$v_{a}$} & \textbf{$t(s)$} \\ \hline
0.3&0.97092&0\\
0.6&0.93931&0\\
0.9&0.68197&0\\
1.2&0.85446&0\\
1.5&0.79951&0\\
1.8&0.73634&0\\
2.1&0.38437&0\\
2.4&0.29738&0\\
\end{tabular}

}
\caption{Datos de la curva de $v_{a}$ para $N = 50$.}
\end{table}
\end{center}
\end{frame}

\subsection{Gráficos}

\begin{frame}
\frametitle{Resultados}
\framesubtitle{Gráfico}
\begin{figure}[H]
        \centering
        \includegraphics[width=0.9\textheight]{example-image}
        \caption{Lorem ipsum dolor sit amet.}
\end{figure}
\end{frame}

\subsection{Animaciones}

\begin{frame}
\frametitle{Resultados}
\framesubtitle{Animación de la prueba $N = ?$}
\begin{figure}[H]
        \centering
        \includegraphics[width=0.9\textheight]{example-image}
        \caption{Lorem ipsum dolor sit amet.}
\end{figure}
\end{frame}

\end{document}